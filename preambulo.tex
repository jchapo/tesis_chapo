% ==================================================================
% PREÁMBULO - PAQUETES Y CONFIGURACIONES
% ==================================================================

% ------------------------------------------------------------------
% PAQUETES BÁSICOS
% ------------------------------------------------------------------
\usepackage[utf8]{inputenc}
\usepackage[spanish,es-tabla]{babel}
\usepackage[T1]{fontenc}
\usepackage{lmodern}

% ------------------------------------------------------------------
% GEOMETRÍA Y MÁRGENES
% ------------------------------------------------------------------
\usepackage[top=2.5cm, bottom=2.5cm, left=3cm, right=2.5cm]{geometry}
\usepackage{setspace}
\onehalfspacing  % Interlineado 1.5

% ------------------------------------------------------------------
% BIBLIOGRAFÍA (IEEE con biblatex)
% ------------------------------------------------------------------
\usepackage[backend=biber,style=ieee,sorting=none]{biblatex}
\addbibresource{referencias.bib}

% ------------------------------------------------------------------
% GRÁFICOS Y FIGURAS
% ------------------------------------------------------------------
\usepackage{graphicx}
\graphicspath{{media/}{figures/}{imagenes/}}
\usepackage{float}
\usepackage{caption}
\usepackage{subcaption}

% ------------------------------------------------------------------
% TABLAS
% ------------------------------------------------------------------
\usepackage{booktabs}
\usepackage{multirow}
\usepackage{array}
\usepackage{longtable}
\usepackage{tabularx}

% ------------------------------------------------------------------
% MATEMÁTICAS
% ------------------------------------------------------------------
\usepackage{amsmath}
\usepackage{amssymb}
\usepackage{amsthm}

% ------------------------------------------------------------------
% HIPERVÍNCULOS
% ------------------------------------------------------------------
\usepackage[hidelinks]{hyperref}
\usepackage{url}
\urlstyle{same}

% ------------------------------------------------------------------
% LISTAS Y ENUMERACIONES
% ------------------------------------------------------------------
\usepackage{enumitem}
\setlist[itemize]{leftmargin=*}
\setlist[enumerate]{leftmargin=*}
\setlist[description]{leftmargin=0cm,style=nextline}

% ------------------------------------------------------------------
% ENCABEZADOS Y PIE DE PÁGINA
% ------------------------------------------------------------------
\usepackage{fancyhdr}
\pagestyle{fancy}
\fancyhf{}
\fancyhead[R]{\thepage}
\fancyhead[L]{\nouppercase{\leftmark}}
\renewcommand{\headrulewidth}{0.4pt}
\fancypagestyle{plain}{%
    \fancyhf{}
    \fancyfoot[C]{\thepage}
    \renewcommand{\headrulewidth}{0pt}
}

% ------------------------------------------------------------------
% CÓDIGO FUENTE (opcional, si incluyes código)
% ------------------------------------------------------------------
\usepackage{listings}
\usepackage{xcolor}

\definecolor{codegreen}{rgb}{0,0.6,0}
\definecolor{codegray}{rgb}{0.5,0.5,0.5}
\definecolor{codepurple}{rgb}{0.58,0,0.82}
\definecolor{backcolour}{rgb}{0.95,0.95,0.92}

\lstdefinestyle{mystyle}{
    backgroundcolor=\color{backcolour},   
    commentstyle=\color{codegreen},
    keywordstyle=\color{magenta},
    numberstyle=\tiny\color{codegray},
    stringstyle=\color{codepurple},
    basicstyle=\ttfamily\footnotesize,
    breakatwhitespace=false,         
    breaklines=true,                 
    captionpos=b,                    
    keepspaces=true,                 
    numbers=left,                    
    numbersep=5pt,                  
    showspaces=false,                
    showstringspaces=false,
    showtabs=false,                  
    tabsize=2
}
\lstset{style=mystyle}

% ------------------------------------------------------------------
% COMANDOS PERSONALIZADOS
% ------------------------------------------------------------------
\newcommand{\universidad}{Pontificia Universidad Católica del Perú}
\newcommand{\facultad}{Facultad de Ciencias e Ingeniería}
\newcommand{\especialidad}{Ingeniería de las Telecomunicaciones}
\newcommand{\autor}{Juan Alfonso Chapoñan Espinoza}
\newcommand{\asesor}{Oscar Antonio Díaz Barriga}
\newcommand{\ciudad}{Lima}
\newcommand{\anio}{2025}

% ------------------------------------------------------------------
% FORMATO DE CAPÍTULOS Y SECCIONES
% ------------------------------------------------------------------
\usepackage{titlesec}

% Formato para capítulos
\titleformat{\chapter}[display]
{\normalfont\huge\bfseries}{\chaptertitlename\ \thechapter}{20pt}{\Huge}
\titlespacing*{\chapter}{0pt}{0pt}{40pt}

% Formato para secciones
\titleformat{\section}
{\normalfont\Large\bfseries}{\thesection}{1em}{}

\titleformat{\subsection}
{\normalfont\large\bfseries}{\thesubsection}{1em}{}

\titleformat{\subsubsection}
{\normalfont\normalsize\bfseries}{\thesubsubsection}{1em}{}

% ------------------------------------------------------------------
% CONFIGURACIÓN DE REFERENCIAS CRUZADAS
% ------------------------------------------------------------------
\usepackage{cleveref}
\crefname{figure}{Figura}{Figuras}
\crefname{table}{Tabla}{Tablas}
\crefname{equation}{Ecuación}{Ecuaciones}
\crefname{chapter}{Capítulo}{Capítulos}
\crefname{section}{Sección}{Secciones}
\crefname{appendix}{Apéndice}{Apéndices}

% ------------------------------------------------------------------
% ESPACIADO DE PÁRRAFOS
% ------------------------------------------------------------------
\setlength{\parskip}{0.5em}
\setlength{\parindent}{0pt}

% ------------------------------------------------------------------
% NOMENCLATURA Y ACRÓNIMOS (opcional)
% ------------------------------------------------------------------
% \usepackage[acronym,toc]{glossaries}
% \makeglossaries

% ------------------------------------------------------------------
% PDF METADATA
% ------------------------------------------------------------------
\hypersetup{
    pdftitle={Plataforma Web y Móvil con IA para Delivery Urbano},
    pdfauthor={Juan Alfonso Chapoñan Espinoza},
    pdfsubject={Tesis de Ingeniería de Telecomunicaciones},
    pdfkeywords={IoT, IA, Delivery, Logística, Computer Vision},
    pdfproducer={LaTeX},
    pdfcreator={pdflatex}
}

% ------------------------------------------------------------------
% IDIOMA PARA ALGORITMOS Y LISTADOS
% ------------------------------------------------------------------
\addto\captionsspanish{
    \renewcommand{\listtablename}{Índice de Tablas}
    \renewcommand{\listfigurename}{Índice de Figuras}
    \renewcommand{\tablename}{Tabla}
    \renewcommand{\figurename}{Figura}
}