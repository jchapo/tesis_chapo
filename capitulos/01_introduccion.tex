% cSpell:language es,en
% ==================================================================
% CAPÍTULO 1: INTRODUCCIÓN
% ==================================================================
\chapter{Introducción}

% ------------------------------------------------------------------
% SECCIÓN: Planteamiento del problema
% ------------------------------------------------------------------
\section{Planteamiento del problema}

En los últimos años, el desarrollo de nuevas tecnologías ha transformado significativamente el panorama de la investigación científica \citep{autor2023}. Este fenómeno presenta tanto oportunidades como desafíos que requieren un análisis profundo.

El problema central de esta investigación se puede formular mediante la siguiente pregunta: ¿Cómo puede optimizarse el proceso de análisis de datos utilizando técnicas de machine learning para mejorar la precisión de los resultados?

% ------------------------------------------------------------------
% SECCIÓN: Justificación
% ------------------------------------------------------------------
\section{Justificación}

La relevancia de este estudio radica en varios aspectos fundamentales:

\begin{itemize}
    \item \textbf{Relevancia teórica}: Contribuye al conocimiento existente sobre métodos de análisis de datos.
    \item \textbf{Relevancia práctica}: Proporciona herramientas aplicables en contextos reales.
    \item \textbf{Relevancia metodológica}: Introduce nuevas técnicas de investigación.
\end{itemize}

Según \citet{otroautor2022}, los avances en este campo pueden tener un impacto significativo en múltiples disciplinas.

% ------------------------------------------------------------------
% SECCIÓN: Objetivos
% ------------------------------------------------------------------
\section{Objetivos}

\subsection{Objetivo general}

Desarrollar e implementar un framework de análisis de datos que integre técnicas de machine learning para mejorar la precisión y eficiencia en el procesamiento de información científica.

\subsection{Objetivos específicos}

\begin{enumerate}
    \item Realizar un análisis exhaustivo del estado del arte en técnicas de machine learning aplicadas al análisis de datos científicos.
    \item Diseñar una arquitectura de software que permita la integración de múltiples algoritmos de aprendizaje automático.
    \item Implementar el framework propuesto y validarlo mediante casos de estudio específicos.
    \item Evaluar el rendimiento del sistema desarrollado comparándolo con métodos tradicionales.
\end{enumerate}

% ------------------------------------------------------------------
% SECCIÓN: Hipótesis
% ------------------------------------------------------------------
\section{Hipótesis}

\textbf{Hipótesis principal}: La implementación de un framework integrado de machine learning mejorará significativamente la precisión del análisis de datos científicos en comparación con métodos tradicionales.

\textbf{Hipótesis específicas}:
\begin{itemize}
    \item H1: El framework propuesto reducirá el tiempo de procesamiento en al menos un 30\%.
    \item H2: La precisión de los resultados se incrementará en un mínimo de 15\%.
    \item H3: La usabilidad del sistema será superior a las herramientas existentes.
\end{itemize}

% ------------------------------------------------------------------
% SECCIÓN: Limitaciones del estudio
% ------------------------------------------------------------------
\section{Limitaciones del estudio}

Este estudio presenta las siguientes limitaciones:

\begin{description}
    \item[Temporal] La investigación se llevó a cabo durante un período de 2 años, lo que puede limitar la generalización de algunos resultados.
    \item[Poblacional] Los casos de estudio se centraron en datos provenientes de un dominio específico.
    \item[Tecnológica] Se utilizaron únicamente herramientas de código abierto disponibles al momento de la investigación.
\end{description}
