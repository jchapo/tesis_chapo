% cSpell:language es,en
% ==================================================================
% ANEXOS DEL CAPÍTULO 3
% ==================================================================

\appendix

% ------------------------------------------------------------------
\chapter{Caracterización detallada de actores del sistema}
\label{anexo:actores}

% ------------------------------------------------------------------

\section*{Actor: Clientes (Emisores)}

\begin{table}[H]
\centering
\begin{tabular}{|p{4cm}|p{10cm}|}
\hline
\textbf{Rol en el sistema} & Usuarios que solicitan el servicio de envío de paquetes \\
\hline
\textbf{Necesidades principales} & Necesitan enviar paquetes de manera rápida y confiable sin invertir tiempo en mediciones manuales. Requieren conocer el estado de sus pedidos en tiempo real y tener transparencia en los costos del servicio. \\
\hline
\textbf{Puntos de dolor actuales} & Actualmente deben medir manualmente los paquetes o proporcionar estimaciones que pueden no corresponder con la realidad. Experimentan falta de trazabilidad durante la entrega y sufren reprogramaciones cuando los paquetes están mal estimados, generando incertidumbre sobre las condiciones del servicio. \\
\hline
\end{tabular}
\end{table}

\section*{Actor: Administradores}

\begin{table}[H]
\centering
\begin{tabular}{|p{4cm}|p{10cm}|}
\hline
\textbf{Rol en el sistema} & Gestores de la operación logística de la empresa \\
\hline
\textbf{Necesidades principales} & Buscan maximizar la eficiencia operativa atendiendo el mayor número de pedidos posible. Necesitan asignar pedidos adecuadamente a motorizados, supervisar entregas en tiempo real y poder rechazar pedidos que excedan la capacidad operativa del servicio. \\
\hline
\textbf{Puntos de dolor actuales} & Enfrentan una gestión manual de pedidos mediante WhatsApp y hojas de Excel que no escala. Carecen de visibilidad en tiempo real de la operación y tienen dificultad para validar las dimensiones reportadas por los clientes, lo que genera pérdida de eficiencia por aceptar paquetes sobredimensionados. \\
\hline
\end{tabular}
\end{table}

\section*{Actor: Motorizados (Repartidores)}

\begin{table}[H]
\centering
\begin{tabular}{|p{4cm}|p{10cm}|}
\hline
\textbf{Rol en el sistema} & Operadores de campo que recogen y entregan paquetes \\
\hline
\textbf{Necesidades principales} & Requieren recibir información clara sobre los pedidos asignados para optimizar el espacio en su mochila o cajas de reparto. Necesitan conocer rutas y ubicaciones con precisión, registrar evidencias de recojo y entrega fácilmente, y evitar viajes innecesarios por paquetes que no pueden transportar. \\
\hline
\textbf{Puntos de dolor actuales} & Experimentan sorpresas frecuentes con paquetes más grandes de lo reportado, lo que genera pérdida de tiempo y combustible en recojos no viables. Carecen de herramientas adecuadas para comunicar el estado del pedido en tiempo real y enfrentan dificultad para gestionar múltiples entregas simultáneas de manera eficiente. \\
\hline
\end{tabular}
\end{table}

\section*{Actor: Destinatarios (Receptores)}

\begin{table}[H]
\centering
\begin{tabular}{|p{4cm}|p{10cm}|}
\hline
\textbf{Rol en el sistema} & Usuarios finales que reciben los paquetes \\
\hline
\textbf{Necesidades principales} & Esperan recibir sus paquetes en el tiempo estimado y conocer cuándo llegará el motorizado. Necesitan tener confianza en el servicio y sentirse informados durante el proceso de entrega. \\
\hline
\textbf{Puntos de dolor actuales} & Sufren reprogramaciones frecuentes debido a problemas logísticos en la cadena de entrega. Experimentan falta de información sobre el estado real de su pedido y quedan insatisfechos cuando el servicio se retrasa sin explicación clara. \\
\hline
\end{tabular}
\end{table}

% ------------------------------------------------------------------
\chapter{Síntesis de hallazgos de la fase de empatía}
\label{anexo:hallazgos}
% ------------------------------------------------------------------

\section{Necesidades identificadas}

\begin{itemize}
    \item \textbf{Automatización de procesos manuales}: La medición de dimensiones de paquetes consume tiempo y genera estimaciones inexactas, mientras que la gestión de pedidos mediante WhatsApp y hojas de cálculo resulta ineficiente para escalar operaciones.
    
    \item \textbf{Visibilidad y trazabilidad en tiempo real}: Clientes requieren conocer el estado de sus envíos, administradores necesitan supervisar la operación completa, y motorizados demandan información clara sobre asignaciones y rutas.
    
    \item \textbf{Optimización de recursos limitados}: Las empresas necesitan soluciones de bajo costo con modelos de pago progresivos, y los motorizados requieren maximizar el aprovechamiento del espacio de transporte para cumplir múltiples entregas eficientemente.
\end{itemize}

\section{Restricciones del contexto}

\begin{itemize}
    \item \textbf{Restricción económica}: Presupuestos limitados para inversión tecnológica inicial y necesidad de evitar costos fijos elevados que comprometan la viabilidad del negocio.
    
    \item \textbf{Restricción técnica}: El uso predominante de dispositivos Android de gama media a baja condiciona las decisiones de diseño hacia arquitecturas que minimicen el procesamiento local. Se implementan estrategias básicas de eficiencia (compresión de imágenes, procesamiento en cloud, uso de SDKs nativos).
    
    \item \textbf{Restricción operativa}: Conectividad variable en diferentes zonas de Lima que requiere estrategias de sincronización y manejo de estados offline para garantizar continuidad del servicio.
    
    \item \textbf{Restricción normativa}: Cumplimiento obligatorio de la Ley N.º 29733 de Protección de Datos Personales en el manejo de ubicaciones, fotografías de evidencia y datos personales de usuarios.
\end{itemize}

\section{Oportunidades de mejora mediante tecnología}

\begin{itemize}
    \item \textbf{Inteligencia artificial para visión por computadora}: Automatización de estimación de dimensiones mediante fotografías con objetos de referencia, eliminando tiempo de medición manual y reduciendo errores humanos.
    
    \item \textbf{Infraestructura cloud escalable (Firebase)}: Sincronización en tiempo real entre dispositivos móviles y dashboard web sin requerir servidores físicos costosos, proporcionando trazabilidad completa de operaciones.
    
    \item \textbf{Notificaciones push (Firebase Cloud Messaging)}: Comunicación instantánea de cambios de estado a todos los actores, mejorando coordinación operativa y experiencia de usuario.
    
    \item \textbf{Geolocalización (Google Maps API)}: Optimización de rutas de entrega y seguimiento en tiempo real de motorizados, aumentando eficiencia del servicio y reduciendo tiempos de entrega.
    
    \item \textbf{Modelo de costos por uso}: Servicios cloud con pago progresional que se alinea con el crecimiento de empresas pequeñas, eliminando barreras de entrada por inversiones iniciales prohibitivas.
\end{itemize}

% ------------------------------------------------------------------
\chapter{Matriz de trazabilidad completa}
\label{anexo:trazabilidad}
% ------------------------------------------------------------------

\begin{longtable}{|p{3.5cm}|p{2.5cm}|p{2.5cm}|p{5.5cm}|}
\caption{Matriz de trazabilidad: Necesidades vs. Requerimientos.} \\
\hline
\textbf{Necesidad identificada} & \textbf{Req. funcionales} & \textbf{Req. no funcionales} & \textbf{Justificación técnica} \\
\hline
\endfirsthead

\multicolumn{4}{c}{\textit{Continuación de la tabla}} \\
\hline
\textbf{Necesidad identificada} & \textbf{Req. funcionales} & \textbf{Req. no funcionales} & \textbf{Justificación técnica} \\
\hline
\endhead

\hline
\endfoot

Automatización de medición de dimensiones & RF2.3, RF2.4 & RNF1.1, RNF6.2, RNF7.1, RNF7.2 & El módulo de IA con Cloud Functions procesa imágenes automáticamente, reduciendo tiempo de medición manual y errores humanos, con compresión optimizada para minimizar consumo de datos. \\
\hline

Trazabilidad en tiempo real de pedidos & RF2.1, RF2.5, RF3.4, RF4.1, RF5.1, RF5.2 & RNF1.2, RNF2.1 & La sincronización en tiempo real de Firestore combinada con notificaciones push proporciona visibilidad completa del estado de pedidos a todos los actores. \\
\hline

Gestión centralizada para administradores & RF4.1, RF4.2, RF4.3 & RNF1.2, RNF4.2, RNF5.1 & El dashboard web con React permite supervisión completa, filtrado eficiente y visualización geográfica de operaciones desde cualquier navegador. \\
\hline

Herramientas operativas para motorizados & RF3.1, RF3.2, RF3.3, RF3.5 & RNF4.1, RNF5.1, RNF6.1 & La aplicación Android nativa optimizada proporciona acceso eficiente a funciones del dispositivo (cámara, GPS) con bajo consumo de batería. \\
\hline

Solución de bajo costo escalable & RF1.1, RF6.1, RF6.2 & RNF2.1, RNF3.1, RNF9.1, RNF9.2 & La arquitectura serverless de Firebase elimina costos de servidores físicos y escala automáticamente con modelo de pago por uso. \\
\hline

Seguridad de información sensible & RF6.1, RF6.2 & RNF3.1, RNF3.2, RNF3.3, RNF7.1-RNF7.5 & Firebase Auth, reglas de seguridad de Firestore, cifrado SSL/TLS y cumplimiento de Ley N.º 29733 protegen datos personales y operativos. \\
\hline

Optimización de espacio de transporte & RF2.3, RF2.4, RF3.5 & RNF6.1 & Las dimensiones precisas permiten a motorizados planificar mejor la capacidad de carga y a administradores asignar pedidos compatibles. \\
\hline

Compatibilidad con dispositivos limitados & RF3.1, RF3.2, RF3.3 & RNF4.1, RNF9.1, RNF9.2 & El diseño considera dispositivos Android de gama media-baja, optimizando procesamiento, memoria y consumo de batería. \\
\hline

\end{longtable}

% ------------------------------------------------------------------
\chapter{Modelo de datos detallado}
\label{anexo:modelo-datos}
% ------------------------------------------------------------------

\section{Colección: USUARIOS}

\textbf{Ruta:} \texttt{usuarios/\{uid\}}

\begin{lstlisting}[language=json,caption={Estructura de documento Usuario}]
{
  uid: string,              // ID unico de Firebase Auth
  email: string,            // Correo electronico
  phone: string,            // Telefono de contacto
  nombre: string,           // Nombre del usuario
  apellido: string,         // Apellido del usuario
  rol: string,              // 'administrador' | 'cliente' | 'motorizado'
  
  // Datos especificos para clientes empresariales (opcional)
  datosEmpresariales: {
    nombreEmpresa: string,
    ruc: string,
    razonSocial: string
  } | null,
  
  // Datos especificos para motorizados (opcional)
  datosMotorizado: {
    estadoServicio: string, // 'disponible' | 'ocupado' | 'desconectado'
    vehiculo: {
      placa: string,
      tipo: string,         // 'motocicleta' | 'bicicleta' | 'auto'
      modelo: string,
      anio: number
    },
    estadisticas: {
      pedidosCompletados: number,
      calificacionPromedio: number,
      tiempoPromedioEntrega: number  // en minutos
    }
  } | null,
  
  // Ultima ubicacion conocida (para motorizados)
  ultimaUbicacionConocida: {
    latitud: number,
    longitud: number,
    timestamp: timestamp,
    distrito: string
  } | null,
  
  // Metadatos
  creadoEn: timestamp,
  actualizadoEn: timestamp,
  activo: boolean,
  ultimaConexion: timestamp
}
\end{lstlisting}

\section{Colección: PEDIDOS}

\textbf{Ruta:} \texttt{pedidos/\{pedidoId\}}

\begin{lstlisting}[language=json,caption={Estructura de documento Pedido (parte 1)}]
{
  id: string,  // ID unico del pedido
  
  // Estructura de asignacion (recojo y entrega pueden ser diferentes)
  asignacion: {
    recojo: {
      estado: string,       // 'pendiente' | 'asignado' | 'completado'
      rutaId: string | null,
      rutaNombre: string | null,
      motorizadoUid: string | null,
      motorizadoNombre: string | null,
      asignadaEn: timestamp | null,
      razonPendiente: string | null
    },
    entrega: {
      // Estructura similar a recojo
    }
  },
  
  // Control de visibilidad por rol
  visibilidad: {
    motorizadoRecojo: boolean,
    motorizadoEntrega: boolean,
    administrador: boolean
  },
  
  // Ciclo operativo del pedido
  cicloOperativo: {
    diaCreacion: string,    // YYYY-MM-DD
    diaAsignado: string,
    cerradoPorAdmin: boolean,
    fechaCierreAdmin: timestamp | null,
    horaLimiteRecojo: timestamp,
    permiteEntregaAntes13h: boolean
  },
  
  // Indices para consultas eficientes
  indices: {
    requiereAsignacionManual: boolean,
    motorizadoRecojoUid: string | null,
    motorizadoEntregaUid: string | null,
    rutaRecojoId: string | null,
    rutaEntregaId: string | null,
    distritoRecojo: string,
    distritoEntrega: string
  }
}
\end{lstlisting}

\begin{lstlisting}[language=json,caption={Estructura de documento Pedido (parte 2)}]
{
  // Informacion del proveedor (quien envia)
  proveedor: {
    uid: string,
    nombre: string,
    correo: string,
    telefono: string,
    direccion: {
      link: string,         // URL de Google Maps
      distrito: string,
      coordenadas: {
        latitud: number,
        longitud: number
      }
    }
  },
  
  // Informacion del destinatario (quien recibe)
  destinatario: {
    nombre: string,
    telefono: string,
    direccion: {
      link: string,
      distrito: string,
      coordenadas: {
        latitud: number,
        longitud: number
      }
    }
  },
  
  // Informacion del paquete
  paquete: {
    detalle: string,        // Descripcion del contenido
    observaciones: string,
    dimensiones: {
      alto: number,         // en centimetros
      ancho: number,
      largo: number,
      volumen: number,      // calculado automaticamente
      peso: number | null,  // opcional
      estimadoPorIA: boolean,
      editadoManualmente: boolean,
      excedeMaximo: boolean,
      confianzaIA: number | null  // 0-1, nivel de confianza del modelo
    },
    fotos: {
      recojo: {
        url: string,
        thumbnail: string,
        timestamp: timestamp,
        procesadaPorIA: boolean
      },
      entrega: {
        url: string,
        thumbnail: string,
        timestamp: timestamp
      },
      comprobantePago: {
        url: string,
        thumbnail: string,
        timestamp: timestamp
      }
    }
  }
}
\end{lstlisting}

\begin{lstlisting}[language=json,caption={Estructura de documento Pedido (parte 3)}]
{
  // Informacion de pago
  pago: {
    seCobra: boolean,
    metodoPago: string,     // 'efectivo' | 'transferencia' | 'yape' | 'plin'
    monto: number,
    comision: number,
    montoTotal: number,
    estadoPago: string,     // 'pendiente' | 'pagado' | 'por_cobrar'
    billeteraUsada: {
      id: string,
      propietarioTipo: string,  // 'proveedor' | 'empresa'
      metodo: string,
      titular: string
    }
  },
  
  // Informacion del motorizado asignado (desnormalizado)
  motorizado: {
    uid: string,
    nombre: string,
    asignadoEn: timestamp,
    aceptadoEn: timestamp | null
  } | null,
  
  // Fechas del ciclo de vida del pedido
  fechas: {
    creacion: timestamp,
    entregaProgramada: timestamp,
    recojo: timestamp | null,
    entrega: timestamp | null,
    anulacion: timestamp | null
  },
  
  // Control de version para actualizaciones concurrentes
  actualizadoEn: timestamp,
  version: number
}
\end{lstlisting}

\section{Colección: BILLETERAS}

\textbf{Ruta:} \texttt{billeteras/\{billeteraId\}}

\begin{lstlisting}[language=json,caption={Estructura de documento Billetera}]
{
  id: string,
  propietario: {
    tipo: string,           // 'proveedor' | 'empresa'
    uid: string | null,     // Si es proveedor, referencia a usuario
    nombreEmpresa: string
  },
  alias: string,            // Nombre descriptivo
  metodo: string,           // 'yape' | 'plin' | 'transferencia_bancaria'
  cuenta: {
    titular: string,
    telefono: string,       // Para Yape/Plin
    numeroDocumento: string | null,  // DNI/RUC
    tipoDocumento: string | null     // 'DNI' | 'RUC'
  },
  qr: {
    urlImagen: string,      // URL de Firebase Storage
    urlImagenThumbnail: string | null,
    hash: string | null     // Para validacion
  },
  config: {
    activo: boolean,
    porDefecto: boolean,
    soloEfectivo: boolean,
    limiteMaximo: number | null,
    comisionPorcentaje: number,
    prioridad: number       // Para ordenar opciones
  },
  verificado: boolean,
  verificadoPor: {
    uid: string,
    nombre: string,
    fecha: timestamp
  },
  creadoEn: timestamp,
  actualizadoEn: timestamp,
  ultimoUso: timestamp | null,
  totalTransacciones: number
}
\end{lstlisting}

\subsection{Índices compuestos necesarios}

\begin{itemize}
    \item \texttt{proveedor.uid + fechas.creacion} (pedidos por cliente)
    \item \texttt{motorizado.uid + fechas.creacion} (pedidos por motorizado)
    \item \texttt{asignacion.recojo.estado + cicloOperativo.diaCreacion} (pedidos pendientes del día)
    \item \texttt{indices.distritoEntrega + fechas.entregaProgramada} (optimización de rutas)
\end{itemize}

% ------------------------------------------------------------------
\chapter{Configuración de seguridad y autenticación}
\label{anexo:seguridad}
% ------------------------------------------------------------------

\section{Estructura de custom claims}

\begin{lstlisting}[language=json,caption={Custom claims en Firebase Authentication}]
{
  rol: 'administrador' | 'cliente' | 'motorizado',
  permisos: ['crear_pedido', 'asignar_motorizado', 'ver_dashboard'],
  empresaId: string | null  // Para clientes empresariales
}
\end{lstlisting}

\section{Reglas de seguridad de Firestore}

\begin{lstlisting}[caption={Firestore Security Rules}]
rules_version = '2';
service cloud.firestore {
  match /databases/{database}/documents {
    // Funcion helper para verificar autenticacion
    function isAuthenticated() {
      return request.auth != null;
    }
    
    // Funcion helper para verificar rol
    function hasRole(role) {
      return isAuthenticated() && 
             request.auth.token.rol == role;
    }
    
    // Usuarios: pueden leer/actualizar su propio perfil
    match /usuarios/{userId} {
      allow read: if isAuthenticated() && 
                     (request.auth.uid == userId || 
                      hasRole('administrador'));
      allow write: if isAuthenticated() && 
                      request.auth.uid == userId;
      allow create: if hasRole('administrador');
    }
    
    // Pedidos: acceso segun rol
    match /pedidos/{pedidoId} {
      allow read: if isAuthenticated() && (
        hasRole('administrador') ||
        (hasRole('cliente') && 
         resource.data.proveedor.uid == request.auth.uid) ||
        (hasRole('motorizado') && 
         resource.data.motorizado.uid == request.auth.uid)
      );
      
      allow create: if isAuthenticated() && 
                       (hasRole('administrador') || 
                        hasRole('cliente'));
      
      allow update: if isAuthenticated() && (
        hasRole('administrador') ||
        (hasRole('motorizado') && 
         resource.data.motorizado.uid == request.auth.uid)
      );
      
      allow delete: if hasRole('administrador');
    }
    
    // Billeteras: solo administradores y propietarios
    match /billeteras/{billeteraId} {
      allow read: if isAuthenticated();
      allow write: if hasRole('administrador') || 
                      resource.data.propietario.uid == request.auth.uid;
    }
  }
}
\end{lstlisting}

\section{Reglas de seguridad de Storage}

\begin{lstlisting}[caption={Firebase Storage Security Rules}]
rules_version = '2';
service firebase.storage {
  match /b/{bucket}/o {
    // Helper functions
    function isAuthenticated() {
      return request.auth != null;
    }
    
    function hasRole(role) {
      return isAuthenticated() && 
             request.auth.token.rol == role;
    }
    
    // Imagenes de pedidos
    match /pedidos/{pedidoId}/{tipo}/{filename} {
      allow read: if isAuthenticated();
      allow write: if isAuthenticated() && (
        hasRole('administrador') ||
        hasRole('cliente') ||
        hasRole('motorizado')
      );
    }
    
    // Fotos de perfil
    match /usuarios/{userId}/{filename} {
      allow read: if isAuthenticated();
      allow write: if isAuthenticated() && 
                      request.auth.uid == userId;
    }
    
    // QR de billeteras
    match /billeteras/{billeteraId}/{filename} {
      allow read: if isAuthenticated();
      allow write: if hasRole('administrador');
    }
  }
}
\end{lstlisting}

% ------------------------------------------------------------------
\chapter{Justificación de puntuaciones en matriz multicriterio}
\label{anexo:justificacion-puntuaciones}
% ------------------------------------------------------------------

\section{Escalabilidad}

\begin{itemize}
    \item \textbf{Firebase (5)}: Escalamiento automático sin configuración, maneja desde 0 hasta millones de usuarios transparentemente.
    
    \item \textbf{AWS (4)}: Excelente escalabilidad pero requiere configuración de Auto Scaling y gestión de capacidad.
    
    \item \textbf{Backend propio (2)}: Requiere planificación manual y redimensionamiento de servidores.
\end{itemize}

\section{Costo}

\begin{itemize}
    \item \textbf{Firebase (5)}: Capa gratuita generosa, ideal para desarrollo de tesis y etapas iniciales de empresas pequeñas.
    
    \item \textbf{AWS (3)}: Capa gratuita limitada a 12 meses, costos variables pero competitivos.
    
    \item \textbf{Backend propio (3)}: Costos fijos predecibles pero presentes desde el inicio.
\end{itemize}

\section{Facilidad de integración}

\begin{itemize}
    \item \textbf{Firebase (5)}: Componentes nativamente integrados, SDKs optimizados, despliegue con un comando.
    
    \item \textbf{AWS (3)}: Requiere integrar múltiples servicios manualmente, configuración compleja.
    
    \item \textbf{Backend propio (2)}: Integración completamente manual de cada componente.
\end{itemize}

\section{Mantenimiento}

\begin{itemize}
    \item \textbf{Firebase (5)}: Gestión de infraestructura completamente delegada a Google.
    
    \item \textbf{AWS (3)}: Gestión parcial, requiere actualización de componentes y monitoreo.
    
    \item \textbf{Backend propio (2)}: Responsabilidad total del mantenimiento, actualizaciones de seguridad, backups.
\end{itemize}

\section{Rendimiento}

\begin{itemize}
    \item \textbf{Firebase (4)}: Excelente para operaciones en tiempo real, latencia baja en sincronización.
    
    \item \textbf{AWS (4)}: Rendimiento configurable según recursos asignados, muy bueno con optimización.
    
    \item \textbf{Backend propio (3)}: Depende del proveedor de hosting y configuración manual.
\end{itemize}

\section{Ecosistema y soporte}

\begin{itemize}
    \item \textbf{Firebase (5)}: Documentación excelente, amplia comunidad, tutoriales especializados.
    
    \item \textbf{AWS (5)}: Documentación exhaustiva, soporte empresarial, ecosistema maduro.
    
    \item \textbf{Backend propio (3)}: Depende de múltiples fuentes, documentación fragmentada.
\end{itemize}